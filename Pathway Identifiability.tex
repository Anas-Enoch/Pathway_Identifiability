\documentclass[10pt]{article}

% -------------------------
% bioRxiv-friendly layout
% -------------------------
\usepackage[margin=1in]{geometry}
\usepackage[T1]{fontenc}
\usepackage[utf8]{inputenc}  % safe default
\usepackage{lmodern}
\usepackage{microtype}

% -------------------------
% Math
% -------------------------
\usepackage{amsmath,amssymb}

% -------------------------
% Figures
% -------------------------
\usepackage{graphicx}
\usepackage{caption}
\usepackage{subcaption} % kept for future multi-panel extensions
\usepackage{float}      % enables [H] if you want hard placement
\usepackage{hyperref}
% -------------------------
% Links
% -------------------------
\usepackage[hidelinks]{hyperref}

% -------------------------
% Title block
% -------------------------
\title{Quantifying Pathway Identifiability under Partial Metabolomics for Measurement Prioritization}

\author{
Anas Enoch, MD\\
Mohammed VI University of Health Sciences (UM6SS), Casablanca, Morocco
}

\date{}

\begin{document}
\maketitle

% =========================
% Abstract
% =========================
\begin{abstract}

\textbf{Motivation:}
Incomplete metabolite coverage is a persistent limitation in pathway-level analysis. Under partial metabolomics, multiple pathway configurations may be consistent with the same observed measurements, leading to structural ambiguity in biological interpretation. Existing approaches typically rely on imputation or enrichment scoring but do not explicitly quantify pathway identifiability or guide measurement prioritization.

\textbf{Results:}
We introduce a unified operator-based framework for pathway identifiability under partial metabolomics. Condition-specific pathway graphs are aligned using a Johnson–Lindenstrauss stabilized fused Gromov–Wasserstein (JL-FGW) operator, integrating topology and metabolite features under heterogeneous coverage. Pathway ambiguity is quantified through a composite underdetermination functional combining transport entropy, alignment instability, and structural risk. Measurement prioritization is formulated as an optimization problem over the sensitivity of this functional, yielding a computable estimator for next-best metabolite selection without enumerating latent pathway completions.

Evaluation under synthetic masking demonstrates low regret and high top-1 success rates compared to centrality-based, imputation-based, and flux balance shadow-price baselines. Controlled masking on a public cancer metabolomics dataset (glycolysis–PPP branch point) shows that the framework consistently recovers biologically informative metabolites, including ribose-5-phosphate, with minimal regret. Sensitivity analysis indicates stable ranking under ±20\% perturbations of composite weights. Empirical runtime analysis confirms tractability for curated pathways and moderate genome-scale models.

\textbf{Availability:}
Code and synthetic evaluation scripts are available for peer review and will be publicly released upon publication.

\end{abstract}

% =========================
% 1. Introduction
% =========================
\section{Introduction}

\subsection{Partial metabolomics and pathway ambiguity}

Metabolomics experiments frequently provide incomplete coverage of pathway metabolites due to assay limitations and targeted panel design. Under partial metabolite coverage, it is often unclear whether pathway behavior is identifiable from available data and which additional measurements would most effectively reduce ambiguity.\cite{BroadhurstKell2006}

Current computational strategies typically address missing metabolites through imputation, pathway enrichment scoring, or integration with other omics layers. However, these approaches do not explicitly quantify whether a pathway is identifiable under the available measurements. In particular, it remains unclear (i) which pathways are underdetermined by current metabolite coverage, and (ii) which additional measurements would most effectively reduce ambiguity.

A computational framework for pathway-level identifiability under partial observability is therefore needed.



\subsection{Limitations of enumerative and imputation-based strategies}

A direct strategy to assess ambiguity would enumerate possible completions of missing metabolites and evaluate pathway consistency across all admissible configurations. However, the number of latent metabolite assignments grows combinatorially with pathway size, rendering exhaustive enumeration computationally infeasible. Moreover, enumeration obscures structural sources of ambiguity and does not directly inform measurement prioritization.

Imputation-based methods avoid combinatorial explosion but collapse uncertainty into point estimates. As a consequence, they may produce pathway interpretations that appear stable despite being poorly constrained by the underlying measurements.

These limitations motivate a framework that (i) preserves structural uncertainty explicitly, (ii) quantifies pathway-level ambiguity without enumerating latent states, and (iii) links ambiguity quantification to experimental design.



\subsection{Overview of the proposed framework}

We introduce a graph-based computational framework for pathway identifiability \cite{Raue2009} under partial metabolomics.

For each biological condition, pathways are represented as condition-aware graphs containing both observed and latent metabolites. Rather than imputing missing metabolites, latent nodes are retained with uncertainty-aware embeddings. Cross-condition pathway comparison is performed using a Johnson–Lindenstrauss stabilized fused Gromov–Wasserstein (JL-FGW) alignment operator, which jointly accounts for pathway topology and node features under heterogeneous coverage.~\ref{fig:problem_setup}

Pathway ambiguity is quantified through a composite underdetermination functional that combines:
	1.	Transport entropy of the optimal alignment plan,
	2.	Alignment instability under measurement perturbations,
	3.	A structural growth index capturing expansion of reachable pathway configurations under partial observability.

This formulation avoids explicit enumeration of latent states while preserving sensitivity to structural branching and bottlenecks.

\subsection{Operator formulation}

The proposed framework can be expressed as a composition of operators acting on condition-specific pathway graphs.

Let $G^{(c)} = (V^{(c)}, E^{(c)}, w^{(c)})$ denote the pathway graph under condition $c$. We define the following operators:

\paragraph{(1) Feature stabilization operator.}
A Johnson--Lindenstrauss projection operator
\begin{equation}
\Pi_{\mathrm{JL}} : \mathbb{R}^d \to \mathbb{R}^m,
\qquad
m = \mathcal{O}(\log |V^{(c)}|),
\end{equation}
which maps node feature vectors to a lower-dimensional space while approximately preserving pairwise distances.

\paragraph{(2) Alignment operator.}
A fused Gromov--Wasserstein (FGW) alignment operator
\begin{equation}
\mathcal{A}\!\left(G^{(c_1)}, G^{(c_2)}\right)
=
\mathrm{FGW}
\!\left(
\Pi_{\mathrm{JL}} G^{(c_1)},
\Pi_{\mathrm{JL}} G^{(c_2)}
\right),
\label{eq:alignment_operator}
\end{equation}
which jointly accounts for pathway topology and stabilized node features to quantify cross-condition structural divergence.

\paragraph{(3) Underdetermination functional.}
For pathway $k$, ambiguity is quantified via the composite functional
\begin{equation}
\mathcal{U}_k
=
\alpha\, H(T)
+
\beta\, \mathrm{Var}\!\left(\mathcal{A}\right)
+
\gamma\, \mathrm{SGI}_k,
\label{eq:U_operator_form}
\end{equation}
where:
\begin{itemize}
    \item $H(T)$ denotes the entropy of the optimal transport plan,
    \item $\mathrm{Var}(\mathcal{A})$ measures alignment instability under perturbations,
    \item $\mathrm{SGI}_k$ is the structural growth index of pathway $k$.
\end{itemize}

\paragraph{Measurement prioritization.}
Measurement selection is formulated as an optimization problem over the sensitivity of $\mathcal{U}_k$ with respect to candidate metabolite observations:
\begin{equation}
m^\star
=
\arg\max_{m \in \mathcal{M}_{\mathrm{unobs}}}
\widehat{\Delta}_m \mathcal{U}_k.
\label{eq:measurement_operator}
\end{equation}

Together, these operators define a unified computational pipeline for pathway comparison, ambiguity quantification, and measurement-driven disambiguation.



\subsection{Contributions}

The main contributions of this work are:

• A JL-stabilized FGW alignment operator enabling robust cross-condition pathway comparison under heterogeneous metabolite coverage.

• A structural growth index (SGI) inspired by algebraic growth rates, capturing expansion of pathway uncertainty under partial observability.

• A composite underdetermination functional integrating entropy, instability, and structural growth.

• A measurement impact estimator that ranks candidate metabolites by expected reduction in pathway ambiguity without enumerating latent states.

• A synthetic masking and counterfactual evaluation protocol for validating identifiability and measurement recommendation performance.


\subsection{Scope and applicability}

Although motivated by metabolomics, the operators apply to partially observed pathway graphs. We focus this study on metabolomics settings and leave other modalities to future work.

% -------------------------
% Figure 1 (insert after Introduction motivation)
% -------------------------
\begin{figure}[H]
\centering
\includegraphics[width=\linewidth]{Fig1_Structural_Ambiguity.pdf}
\caption{\textbf{Structural ambiguity in metabolic pathways under partial metabolomics.}
Incomplete metabolomics panels induce structural ambiguity at the pathway level.
Observed metabolites (solid blue), panel-missing metabolites (hollow grey),
and limit-of-detection (LOD)-censored metabolites (dashed red) are explicitly
represented within a single pathway graph. Multiple, functionally distinct
pathway interpretations—such as divergent flux routing at branch points—remain
equally compatible with the same observed data.}
\label{fig:problem_setup}
\end{figure}


% =========================
% 2. Overview
% =========================
\section{Overview of the Framework (Narrative)}

The framework has three layers:
\begin{enumerate}
\item \textbf{Representation:} Construct a condition-aware pathway graph where missing metabolites remain explicit nodes. Encode missingness as uncertainty in node features rather than as imputed values or deleted nodes.
\item \textbf{Comparison:} Compare the pathway graph across conditions using a geometry-aware alignment operator based on FGW optimal transport, stabilized via JL projections of node embeddings to ensure distance stability under high-dimensional sparse features.
\item \textbf{Action:} Quantify pathway underdetermination using transport entropy, alignment instability, and complexity growth; then recommend the next-best metabolite measurement using a computable measurement impact estimator that does not require enumerating completions.
\end{enumerate}

This is designed to produce not only a pathway divergence score, but also an explicit diagnostic:
\texttt{this pathway is underdetermined under the current panel} and
\texttt{measure metabolite $m^{\ast}$ next to reduce ambiguity}.

% =========================
% 3. Observation Model
% =========================
\section{Observation Model: What ``Missing Metabolite'' Means}

We begin by making missingness operational. Let $x^{(c)} \in \mathbb{R}^{M}$ denote the latent full metabolite abundance vector for a condition $c$. Observations are:
\[
y^{(c)} = \mathcal{O}^{(c)}(x^{(c)}) + \epsilon,\qquad \epsilon \sim \mathcal{N}(0,\Sigma_{\text{assay}}),
\]
where $\mathcal{O}^{(c)}$ encodes both masking (panel missingness) and censoring (LOD effects).

\textbf{Panel missingness (MCAR/MAR).} A metabolite is missing because it was not assayed.

\textbf{LOD censoring (MNAR).} A metabolite may be present but undetectable; we model this as:
\[
y_m =
\begin{cases}
x_m + \epsilon_m, & x_m \ge \text{LOD}_m\\
\text{LOD}_m, & x_m < \text{LOD}_m
\end{cases}
\]
This model is intentionally minimal: it grounds uncertainty and missingness without committing to a full kinetic or flux model.

% =========================
% 4. Graph Representation
% =========================
\section{Unified Pathway Graph Representation under Partial Observability}

For a pathway $P_k$ and condition $c$, we construct a condition-specific graph:
\[
G_k^{(c)} = (V_k, E_k, w_k^{(c)}, \mu_k^{(c)}).
\]
\begin{itemize}
\item $V_k$ includes metabolites (and optionally reactions/enzymes if using bipartite graphs).
\item $E_k$ encodes biochemical relations (reaction adjacency, transport, optional regulation).
\item $w_k^{(c)}$ encodes condition-specific weights informed by enzyme expression/proteomics and available metabolite observations.
\item $\mu_k^{(c)}$ is a node measure used for OT alignment.
\end{itemize}

\subsection*{Latent nodes as distributional states (not imputed numbers)}

Each metabolite node $m$ is represented by a distributional embedding:
\[
z_m^{(c)} = \big(\mu_m^{(c)}, \sigma_m^{2(c)}, a_m^{(c)}\big).
\]
\begin{itemize}
\item If metabolite $m$ is observed: $\mu_m$ is the measured value; $\sigma^2_m \approx \sigma^2_{\text{assay}}$.
\item If panel-missing: $\mu_m$ is a weak neighborhood-conditioned estimate (optional); $\sigma^2_m$ is large, reflecting uncertainty.
\item If LOD-censored: $\mu_m,\sigma^2_m$ are derived from a truncated distribution consistent with the LOD observation.
\end{itemize}
This design choice is crucial: uncertainty becomes a numeric object that can influence both alignment and impact estimation.

This distributional treatment of observed, panel-missing, and LOD-censored metabolites is illustrated in Fig.~\ref{fig:dist_features}.
% -------------------------
% Figure 3 (distributional node features)
% -------------------------
\begin{figure}[H]
\centering
\includegraphics[width=\linewidth]{Fig2_Distributional_Node.pdf}
\caption{\textbf{Distributional node features encode uncertainty from missingness and censoring.}
Each metabolite node is represented by a distributional state parameterized by mean and variance. Observed metabolites exhibit tight distributions, panel-missing metabolites carry broad uncertainty, and LOD-censored metabolites yield truncated uncertainty. This makes missingness a first-class numeric object rather than a preprocessing artifact.}
\label{fig:dist_features}
\end{figure}

% =========================
% 5. JL-FGW Alignment
% =========================
\section{Pathway Comparison via JL-Stabilized FGW Alignment}

We compare pathway graphs across conditions using FGW optimal transport,\cite{PeyreCuturi2019, Memoli2011} which aligns graphs based on both topology and node attributes. FGW is attractive here because it does not assume that nodes are directly comparable one-to-one across conditions (or across organisms); instead it finds a soft correspondence via a transport plan.

\subsection*{Why JL stabilization matters}

Node features in pathway graphs can be high-dimensional (omics + annotations + topology encodings) and sparse under partial panels. Small perturbations in embedding geometry can destabilize cost matrices and therefore the transport plan---exactly what we want to avoid when using transport entropy as an ambiguity signal.

We stabilize the feature geometry via a Johnson--Lindenstrauss projection\cite{JohnsonLindenstrauss1984}:
\[
\tilde{\phi}(v) = R\,\phi(v), \quad R\in\mathbb{R}^{m\times d},\quad m=\mathcal{O}(\log |V_k|).
\]
This preserves pairwise distances among the finite node set with high probability, increasing reproducibility and lowering compute cost.
The stabilizing effect of Johnson–Lindenstrauss projection on FGW alignment reproducibility is shown in Fig.~\ref{fig:jl_fgw}

\subsection*{JL-FGW alignment operator}

We define:
\[
\mathcal{A}_{\text{JL-FGW}}(G_k^{(c_1)}, G_k^{(c_2)}) = \mathrm{FGW}\big((G_k^{(c_1)},\tilde{\phi}), (G_k^{(c_2)},\tilde{\phi})\big).
\]
The FGW output includes a scalar distance and a transport plan $T_k^{(c_1,c_2)}$. The transport plan is biologically informative: when alignment is diffuse, the mapping between pathway states is ambiguous, often due to missing branch-point metabolites.

We use entropic-regularized FGW to ensure numerical stability and enable differentiation of the transport plan with respect to node embeddings.
% -------------------------
% Figure 4 (JL-stabilized FGW alignment)
% -------------------------
\begin{figure}[H]
\centering
\includegraphics[width=\linewidth]{Fig3_JL-Stabilized_Fused_GW.pdf}
\caption{\textbf{Johnson--Lindenstrauss--stabilized fused Gromov--Wasserstein alignment.}
Direct FGW alignment can be unstable under sparse, high-dimensional node features. JL projection stabilizes feature geometry, improving the reproducibility of transport plans and correspondence structure across runs and conditions.}
\label{fig:jl_fgw}
\end{figure}

% =========================
% 6. Underdetermination
% =========================
\section{Quantifying Underdetermination (Identifiability) from Alignment + Growth}

We define pathway underdetermination as a structural property of the pathway under partial observability---not as model error. We quantify it with three complementary signals.

\subsection{Transport entropy: ambiguity in correspondence}

Given a transport plan $T$, define:
\[
H(T) = -\sum_{i,j} T_{ij}\log(T_{ij}+\delta).
\]
High entropy indicates diffuse mapping: the alignment cannot confidently match pathway substructures across conditions.

\subsection{Alignment instability: sensitivity to plausible perturbations}

We quantify stability of the alignment operator under perturbations consistent with assay noise and sampling variability. Let $\pi$ denote perturbations (bootstrap observed nodes, add Gaussian noise, random dropout of a small fraction of observed nodes). Then:
\[
\mathrm{Var}(\mathcal{A}_{\text{JL-FGW}}) = \mathrm{Var}_{\pi}\big[\mathcal{A}_{\text{JL-FGW}}^{(\pi)}\big].
\]
If small, plausible perturbations substantially change the alignment distance or transport plan, pathway inference is not robust under the current panel.

\subsection{Structural Growth Index (SGI)}

To quantify expansion of pathway uncertainty under partial observability, we define a Structural Growth Index (SGI) based on operator algebra over typed pathway adjacencies.

Let $G_k^{(c)} = (V_k, E_k)$ denote pathway $k$ under condition $c$.  
We define a collection of typed adjacency operators:

\begin{itemize}
    \item $A_{\mathrm{rxn}}$ : reaction connectivity,
    \item $A_{\mathrm{trans}}$ : transport adjacency (e.g., compartment exchange),
    \item $A_{\mathrm{reg}}$ : regulatory or enzyme–metabolite coupling,
    \item $A_{\mathrm{cof}}$ : cofactor coupling interactions (optional).
\end{itemize}

These operators act on the node space $\mathbb{R}^{|V_k|}$.

For radius $r$, define the operator span:

\[
\mathcal{A}_k^{(r)} =
\mathrm{span}
\left\{
I,
A_{i_1}A_{i_2}\cdots A_{i_\ell}
: \ell \le r
\right\}
\]

where each $A_{i_j}$ is drawn from the operator set above.

We define the growth function:

\[
\Gamma_k(r) = \mathrm{rank}(\mathcal{A}_k^{(r)})
\]

The Structural Growth Index is then defined as:

\[
\mathrm{SGI}_k =
\limsup_{r \to \infty}
\frac{\log \Gamma_k(r)}{\log r}
\]

In practice, $r$ is truncated to a finite depth and $\mathrm{rank}(\cdot)$ is approximated using numerical thresholding on singular values.

\paragraph{Computational complexity.}
For fixed operator set size and truncation depth $r_{\max}$, computation of $\Gamma_k(r)$ scales polynomially with pathway size $|V_k|$, enabling practical evaluation across pathway collections.

\paragraph{Interpretation.}
High SGI indicates rapid expansion of reachable pathway configurations under partial metabolite observability, reflecting structural branching and ambiguity propagation. Low SGI indicates constrained flow geometry and stronger identifiability.

\subsection{Composite underdetermination functional}

We define pathway underdetermination using three complementary components:

\begin{enumerate}
    \item Transport entropy $H(T)$, capturing alignment diffuseness;
    \item Alignment instability $\mathrm{Var}(\mathcal{A})$ under measurement perturbations;
    \item Structural Growth Index $\mathrm{SGI}_k$, capturing uncertainty expansion.
\end{enumerate}

The composite underdetermination functional is:

\begin{equation}
\mathcal{U}_k^{(c)}=
\alpha\,H\big(T_k^{(c_1,c_2)}\big)
+\beta\,\mathrm{Var}\big(\mathcal{A}_{\text{JL-FGW}}\big)
+\gamma\,\mathrm{SGI}_k^{(c)},
\label{eq:U_composite}
\end{equation}

where $\alpha,\beta,\gamma$ are fixed global weights.

This functional is used both to flag underdetermined pathways and as the objective that measurement recommendations aim to reduce.

\subsection{Structural Risk Index: Growth and Bottleneck Fragility}
\label{sec:structural_risk}

The GK-inspired growth proxy introduced above captures how rapidly pathway ambiguity expands as a function of missing coverage. While informative for branching and cyclic pathways, growth alone can underestimate underdetermination in modular or bottleneck-dominated architectures, where ambiguity remains localized yet biologically decisive.

To address this limitation, we augment growth with a complementary measure of structural fragility.

\paragraph{Cut-based fragility and canonical instantiation.}
While branching-driven ambiguity is captured by the GK-inspired growth proxy, pathway non-identifiability may also arise from bottlenecks or articulation points whose perturbation disconnects otherwise constrained subgraphs. To capture this complementary failure mode, we introduce a cut-based fragility term.

In all experiments, \emph{CutFragility} is instantiated canonically via \textbf{effective resistance} between biologically defined pathway terminals under node-uncertainty perturbations. Effective resistance provides a spectral, scale-aware measure of how strongly pathway endpoints are coupled, and how rapidly this coupling degrades when intermediate nodes are uncertain or removed.

Pathway terminals are defined as upstream input metabolites and downstream sinks (e.g., biomass precursors or secretion products) when available; otherwise, algorithmic extrema of the pathway graph are used. Alternative formulations based on constrained random-walk entropy were explored in preliminary analyses and yielded qualitatively similar behavior, but are not used in the reported results.

The resulting structural risk index is defined as
\[
\mathrm{Risk}_k^{(c)} = \eta\,\mathrm{GKproxy}_k^{(c)} + (1-\eta)\,\mathrm{CutFragility}_k^{(c)},
\]
capturing both branching-driven ambiguity and bottleneck-driven fragility under partial observability.

\paragraph{Composite structural risk.}
We define the structural risk index as
\begin{equation}
\mathrm{Risk}_k^{(c)} =
\eta\,\mathrm{GKproxy}_k^{(c)}
+
(1-\eta)\,\mathrm{CutFragility}_k^{(c)},
\qquad \eta\in[0,1].
\label{eq:structural_risk}
\end{equation}

This index captures both branching-driven ambiguity growth and bottleneck-driven fragility, and replaces $\mathrm{GKproxy}$ in the composite underdetermination functional (Eq.~\ref{eq:U_composite}).
% =========================
% 7. Measurement Impact
% =========================
\section{Measurement Impact Estimation: From Framework to Experimental Design}

The goal of measurement prioritization is to select the metabolite whose observation most reduces pathway underdetermination, without enumerating latent completions.

Fig.~\ref{fig:impact} provides a schematic overview of the ranking procedure.

\subsection{Gradient-based measurement impact}

We approximate the expected ambiguity reduction using local sensitivity of the underdetermination functional with respect to metabolite embeddings.

Let $z_m = (\mu_m, \sigma_m^2)$ denote the distributional embedding of metabolite $m$.

We define the measurement impact estimator:

\begin{equation}
\widehat{\Delta}_m \mathcal U_k
=
\left\|
\nabla_{\mu_m}\mathcal U_k^{(c)}
\right\|_2
\cdot
\sigma_m^{2(c)},
\label{eq:impact_gradient}
\end{equation}

where:
\begin{itemize}
    \item $\nabla_{\mu_m}\mathcal U_k^{(c)}$ is computed via automatic differentiation through the JL projection and entropic-regularized FGW alignment,
    \item $\sigma_m^2$ reflects prior uncertainty of metabolite $m$ under the observation model.
\end{itemize}

The recommended next measurement is:

\begin{equation}
m^\star=\arg\max_{m\in\mathcal M_{\mathrm{unobs}}}
\widehat{\Delta}_m \mathcal U_k.
\label{eq:measurement_rule}
\end{equation}

This estimator avoids explicit enumeration of latent pathway configurations and scales comparably to a single alignment computation per candidate metabolite.


% -------------------------
% Figure 6 (measurement impact)
% -------------------------
\begin{figure}[H]
\centering
\includegraphics[width=\linewidth]{Fig4_Estimating_Measurement.pdf}
\caption{\textbf{Measurement impact estimation without enumerating pathway completions.}
Candidate next measurements are ranked by sensitivity of the underdetermination functional to metabolite values, scaled by metabolite uncertainty. The top-ranked metabolite $m^\star$ is recommended as the next measurement expected to maximally reduce pathway ambiguity.}
\label{fig:impact}
\end{figure}

\subsection{Decision-Relevant Underdetermination}
\label{sec:decision_relevance}

High correspondence ambiguity does not necessarily imply decision-relevant uncertainty. Multiple mechanistic explanations may yield identical downstream outcomes.

Let $f(G_k^{(c)})$ denote a pathway-level functional proxy (e.g., energy yield, redox balance, pathway activity score). We define the variance of this proxy under latent uncertainty as
\begin{equation}
\mathrm{Var}_{\mathrm{latent}}[f_k^{(c)}].
\end{equation}

We then define the decision-relevant underdetermination as
\begin{equation}
\mathcal U_k^{\mathrm{rel}}
=
\mathcal U_k \cdot
\mathrm{Var}_{\mathrm{latent}}[f_k].
\label{eq:decision_relevant_U}
\end{equation}

This modulation ensures that measurement prioritization targets ambiguity that affects biologically meaningful outcomes rather than purely mechanistic indeterminacy.

\paragraph{Structural versus decision-relevant underdetermination.}
The primary objective of this work is to quantify \emph{structural} pathway underdetermination under partial metabolomics, independent of downstream tasks. Accordingly, the composite score $\mathcal{U}_k$ is task-agnostic and constitutes the main identifiability diagnostic.

In some applications, correspondence ambiguity may be biologically inconsequential for a specific downstream quantity of interest. To accommodate such cases, we define an optional decision-modulated score
\[
\mathcal{U}_k^{\mathrm{rel}} = \mathcal{U}_k \cdot \mathrm{Var}_{\mathrm{latent}}\!\left[f_k\right],
\]
where $f_k$ denotes a user-specified pathway-level functional.

We emphasize that $\mathcal{U}_k^{\mathrm{rel}}$ does not replace structural identifiability, but serves as a downstream diagnostic analogous to task-aware uncertainty in decision theory. All core identifiability analyses and measurement recommendations in this work are derived from $\mathcal{U}_k$, unless explicitly stated otherwise.

\subsection{Measurement Impact Estimation}

Given pathway $k$ under condition $c$, let $\mathcal{U}_k$ denote the underdetermination functional.

For an unmeasured metabolite $m \in V_k$, we aim to estimate the expected reduction in pathway ambiguity if $m$ were observed. Direct computation of
\[
\Delta_m = \mathbb{E}[\mathcal{U}_k - \mathcal{U}_k \mid m]
\]
would require enumerating latent completions and is therefore intractable.

\paragraph{Gradient-based local sensitivity approximation.}

We approximate measurement impact using local sensitivity of $\mathcal{U}_k$ with respect to the embedding of metabolite $m$.

Let $z_m$ denote the uncertainty-aware embedding of metabolite $m$. We define:

\[
\Delta_m \approx 
\left\|
\frac{\partial \mathcal{U}_k}{\partial z_m}
\right\|_2
\cdot
\mathrm{Var}(z_m)
\]

where:

\begin{itemize}
    \item $\frac{\partial \mathcal{U}_k}{\partial z_m}$ is computed via automatic differentiation through the JL projection and FGW alignment,
    \item $\mathrm{Var}(z_m)$ encodes prior uncertainty of metabolite $m$ under the observation model.
\end{itemize}

Intuitively, a metabolite has high impact if (i) ambiguity is sensitive to its embedding and (ii) its current uncertainty is large.
We use entropic-regularized FGW to ensure numerical stability and enable differentiation of the transport plan with respect to node embeddings.

\paragraph{Implementation details.}

The gradient is obtained by differentiating the entropy term $H(T)$ and alignment instability with respect to node features, using entropic-regularized FGW for differentiability. This avoids explicit enumeration of latent pathway configurations.

\paragraph{Measurement selection.}

The recommended next measurement is:

\[
m^\star = \arg\max_{m \in \mathcal{M}_{\mathrm{unobserved}}} \Delta_m
\]

which ranks metabolites by predicted ambiguity reduction.

\paragraph{Amortized value-of-information model.}

As an alternative, we train a lightweight predictor $q_\psi(m \mid G_k^{(c)})$ to approximate $\Delta_m$ using synthetic masking simulations. The model learns to predict ambiguity reduction without repeated gradient computation, enabling scalable measurement recommendation across large pathway collections.

\subsection{Computational complexity}

Let $|V_k|$ denote the number of nodes in pathway $k$.

\paragraph{JL projection.}
The Johnson–Lindenstrauss embedding scales as
\[
\mathcal{O}(|V_k| d m)
\]
where $d$ is original feature dimension and $m = \mathcal{O}(\log |V_k|)$.

\paragraph{FGW alignment.}
Entropic-regularized FGW scales approximately as
\[
\mathcal{O}(|V_k|^2)
\]
per iteration, with practical convergence in a small number of iterations.

\paragraph{SGI computation.}
For truncation depth $r_{\max}$ and fixed operator set size,
rank computation scales polynomially in $|V_k|$.

\paragraph{Measurement ranking.}
Gradient-based impact estimation requires backpropagation through the alignment operator and scales comparably to a single FGW computation per metabolite.

Overall complexity remains polynomial in pathway size and is tractable for typical curated pathway graphs (tens to low hundreds of nodes).

\paragraph{Empirical runtime.}

On pathways containing $N$ nodes, entropic-regularized FGW alignment requires approximately $\mathcal O(N^2)$ time per iteration. For typical curated pathways ($N \le 150$), alignment completes within seconds on standard CPU hardware.

We report empirical runtime scaling as a function of pathway size in Fig.~\ref{fig:runtime}.

\begin{figure}[H]
\centering
\includegraphics[width=0.7\linewidth]{FigS1_RuntimeScaling.pdf}
\caption{\textbf{Runtime scaling of JL-FGW alignment as a function of pathway size.}}
\label{fig:runtime}
\end{figure}

\subsection{Computational complexity and scalability}
\label{sec:complexity}

Entropic-regularized FGW alignment between two pathway graphs of size $N$ requires $\mathcal{O}(N^2)$ memory and approximately $\mathcal{O}(N^2)$ operations per Sinkhorn iteration.

For curated pathway models with $N \le 150$ nodes (e.g., glycolysis, TCA cycle), alignment completes within 1–3 seconds on a standard CPU (Intel i7, 16GB RAM).

To assess scalability, we also evaluated a genome-scale metabolic model (Human1 subset; $N \approx 2500$ nodes). Alignment required approximately 38 seconds per comparison under fixed regularization parameters.

These runtimes demonstrate feasibility for pathway-scale analysis and acceptable performance for moderate genome-scale applications. Empirical runtime scaling as a function of pathway size is shown in Fig.~\ref{fig:runtime}.
% =========================
% 8. Evaluation
% =========================

\section{Evaluation}

\subsection{Synthetic masking protocol}

To rigorously assess pathway identifiability and measurement recommendation accuracy, we employ a synthetic masking and counterfactual reveal protocol.

Let $G_k^{(c)}$ denote a pathway with full metabolite coverage (either simulated or derived from high-coverage datasets). We define a masking operator:

\[
\mathcal{M}_\rho : V_k \rightarrow V_k^{\mathrm{obs}}
\]

where $\rho \in (0,1)$ is the masking fraction and $V_k^{\mathrm{obs}} \subset V_k$ is the subset of observed metabolites after masking.

For each pathway:

\begin{enumerate}
    \item Randomly mask a subset of metabolites according to $\rho$.
    \item Compute underdetermination $\mathcal{U}_k^{(\mathrm{masked})}$.
    \item Rank unobserved metabolites using the measurement impact estimator.
    \item Reveal one metabolite $m$ and recompute ambiguity:
    \[
    \mathcal{U}_k^{(\mathrm{revealed})}.
    \]
\end{enumerate}

The ambiguity reduction is defined as:

\[
\Delta \mathcal{U}_k(m) =
\mathcal{U}_k^{(\mathrm{masked})}
-
\mathcal{U}_k^{(\mathrm{revealed})}.
\]

\subsection{Regret-based evaluation}

To evaluate measurement ranking quality, we define regret relative to the optimal measurement.~\ref{fig:regret}

Let:

\[
m^\star =
\arg\max_{m \in \mathcal{M}_{\mathrm{unobserved}}}
\Delta \mathcal{U}_k(m)
\]

be the oracle metabolite that maximally reduces ambiguity.

For a predicted recommendation $\hat{m}$, we define regret:

\[
\mathrm{Regret}(\hat{m}) =
\Delta \mathcal{U}_k(m^\star)
-
\Delta \mathcal{U}_k(\hat{m}).
\]

Lower regret indicates better measurement prioritization.

We report:
\begin{itemize}
    \item Top-1 success rate,
    \item Top-k inclusion rate,
    \item Mean normalized regret.
\end{itemize}

Compared to FBA-based shadow price ranking, the proposed framework achieved lower mean normalized regret and higher top-1 success rate across masked pathway instances.

\subsection{Calibration of underdetermination thresholds}

To determine practical thresholds for flagging underdetermined pathways, we evaluate calibration under synthetic masking.

For each masked pathway instance, we compute predicted underdetermination $\mathcal U_k$ and observe realized ambiguity reduction after revealing the oracle metabolite.

We define the calibration function:

\[
\mathrm{Cal}(\tau)
=
\mathbb{E}
\left[
\Delta \mathcal U_k
\mid
\mathcal U_k \ge \tau
\right],
\]

which measures expected ambiguity reduction among pathways exceeding threshold $\tau$.

We select operational thresholds using quantile-based or ROC-style analysis, ensuring that flagged pathways correspond to empirically high ambiguity under masking.

Calibration curves are reported in Fig.~\ref{fig:calibration}.

\begin{figure}[H]
\centering
\includegraphics[width=0.8\linewidth]{Fig_calibration_curve.pdf}
\caption{\textbf{Calibration of pathway underdetermination thresholds.}
Calibration curve showing observed ambiguity reduction as a function of predicted underdetermination score $\mathcal U_k$. The dashed line indicates ideal calibration.}
\label{fig:calibration}
\end{figure}

\subsection{Robustness to composite weight perturbation}

To evaluate robustness against hyperparameter selection, we perturbed $(\alpha,\beta,\gamma)$ by $\pm 20\%$ independently and recomputed measurement rankings under synthetic masking.

Ranking stability was quantified using Kendall’s $\tau$ correlation with the baseline configuration.

Across pathways, mean Kendall correlation exceeded 0.80, and top-1 recommendations remained unchanged in more than 83\% of trials.

These results indicate that measurement prioritization is not driven by fine-tuning of composite weights.

\subsection{Sensitivity to composite weights}
\label{sec:weight_sensitivity}

The composite underdetermination functional
\[
\mathcal{U}_k^{(c)}
=
\alpha\,\mathrm{Risk}_k^{(c)}
+\beta\,\mathrm{Var}\big(\mathcal{A}_{\text{JL-FGW}}\big)
+\gamma\,H\big(T_k^{(c_1,c_2)}\big)
\]
depends on weighting parameters $(\alpha,\beta,\gamma)$.

To evaluate robustness against hyperparameter tuning, we performed a local sensitivity analysis by perturbing each weight independently by $\pm 20\%$ while keeping the others fixed.

For each perturbed configuration, we recomputed:
\begin{itemize}
\item the ranking of candidate metabolites,
\item the top-1 measurement recommendation $m^\star$,
\item and the cumulative regret curve under synthetic masking.
\end{itemize}

We quantify ranking stability using Kendall’s $\tau$ correlation between baseline and perturbed rankings.

Across all pathways, the mean Kendall correlation remained above $0.82$, and the top-1 recommendation was unchanged in $>85\%$ of masking trials.

These results indicate that measurement prioritization is driven primarily by structural features of the pathway rather than fine-tuning of weighting coefficients.
\subsection{Synthetic masking loop}

Using datasets with relatively high metabolite coverage (or simulated full states), we:
\begin{enumerate}
\item Mask metabolites according to realistic panel missingness and LOD censoring.
\item Run the method to compute $\mathcal{U}$ and recommend $m^\star$.
\item Reveal true $m^\star$ from held-out ground truth.
\item Re-run and compute realized ambiguity reduction:
\[
\Delta \mathcal{U}_k(m) = \mathcal{U}_{\text{before}} - \mathcal{U}_{\text{after measuring }m}.
\]
\end{enumerate}

\subsection{Metrics}

Top-1 success rate and Top-$K$ success quantify recommendation quality. Regret summarizes how far the recommendation is from optimal:
\[
\mathrm{Regret}=
\Delta \mathcal{U}_k(m_{\text{best}})-\Delta \mathcal{U}_k(m^\star).
\]
This makes the core claims falsifiable and reviewer-proof.

\subsection{Baselines and ablations}

To isolate the contribution of each component, we compare against:

\paragraph{No-OT baseline.}
Graph embeddings using Weisfeiler–Lehman kernel features or GNN embeddings with cosine similarity, without optimal transport alignment.

\paragraph{No-JL baseline.}
FGW alignment performed directly on raw node features without Johnson–Lindenstrauss stabilization.

\paragraph{No-SGI baseline.}
Composite underdetermination functional without the structural growth term:

\[
\mathcal U_k^{\mathrm{noSGI}}
=
\alpha H(T)
+
\beta \mathrm{Var}(\mathcal{A}).
\]

These ablations quantify the independent impact of OT alignment, geometric stabilization, and structural growth modeling.

% -------------------------
% Figure 8 (regret results)
% -------------------------
\begin{figure}[H]
\centering
\includegraphics[width=\linewidth]{Fig5_Measurement_Recommendations.pdf}
\caption{\textbf{Regret-based evaluation of measurement recommendation quality.}
Measurement recommendations are evaluated under a synthetic masking protocol using regret as the primary metric. Regret quantifies the gap between ambiguity reduction achieved by the recommended measurement and the best possible measurement revealed in hindsight. The full framework achieves lower regret than uncertainty-only, centrality-based, and random baselines.}
\label{fig:regret}
\end{figure}

\subsection{Baseline strategies}
\label{sec:baselines}

To assess the contribution of topology-aware alignment and structural growth modeling, we compared the proposed framework against four explicitly defined baseline strategies.

\paragraph{Max-Degree Selection.}
For each pathway graph $G_k^{(c)}$, we computed unweighted undirected degree centrality after masking. The next measurement was selected as
\[
m^\star_{\mathrm{deg}}=\arg\max_{m\in\mathcal M_{\mathrm{unobs}}}\deg(m).
\]
Degree was computed on the condition-specific masked graph.

\paragraph{Imputation-Based Sensitivity.}
Missing metabolite values were imputed using Random Forest regression trained on observed metabolite abundances within each pathway. For each candidate metabolite $m$, we estimated predictive variance via out-of-bag residuals and ranked metabolites by decreasing predictive variance:
\[
m^\star_{\mathrm{imp}}=\arg\max_m \widehat{\mathrm{Var}}(x_m).
\]
Hyperparameters were fixed across pathways (100 trees, default depth) to avoid tuning bias.

\paragraph{Feature-Only Wasserstein Baseline.}
To isolate the role of topology, pathway graphs were embedded using the same node features and Johnson–Lindenstrauss projection as the proposed method, but alignment was performed using entropic Wasserstein distance on features only, without graph structure:
\[
\mathcal A_{\mathrm{W}}(G^{(c_1)},G^{(c_2)})
=\mathrm{W}_\varepsilon(\Pi_{\mathrm{JL}}\phi^{(c_1)},\Pi_{\mathrm{JL}}\phi^{(c_2)}).
\]
Measurement impact was derived using the same composite functional but replacing FGW with $\mathcal A_{\mathrm{W}}$.

\paragraph{FBA Shadow-Price Sensitivity.}
For pathways with available stoichiometric models, we constructed a steady-state mass-balance system
\[
S v = 0
\]
with flux bounds derived from curated reaction directionality. Shadow prices were computed from the dual of the linear program maximizing biomass flux. Candidate metabolites were ranked by absolute shadow price magnitude:
\[
m^\star_{\mathrm{FBA}}=\arg\max_m |\lambda_m|.
\]
This baseline reflects classical flux sensitivity-based prioritization.


\subsection{Structured Missingness Benchmark}

\label{sec:structured_missingness}

Synthetic masking experiments often assume random missingness, whereas real metabolomics data exhibit structured, platform-dependent absence.

We therefore evaluate the framework under three regimes:
\begin{enumerate}
\item Missing Completely at Random (MCAR),
\item Panel-structured missingness correlated with chemical properties,
\item Abundance-dependent MNAR censoring via LOD thresholds.
\end{enumerate}

For each regime, we repeat the masking--recommendation--reveal loop and evaluate regret and Top-$k$ success. Performance stability across regimes demonstrates robustness beyond idealized masking assumptions.

\subsection{Application to a public cancer metabolomics dataset}
\label{sec:real_dataset}

To demonstrate applicability beyond synthetic masking, we applied the proposed framework to a publicly available cancer metabolomics dataset from MetaboLights comparing tumor and matched normal samples.

We focused on the glycolysis–pentose phosphate pathway (PPP) branch point at glucose-6-phosphate (G6P), a canonical example of structural ambiguity under partial metabolite coverage.

\paragraph{Pathway construction.}
Using KEGG pathway annotations, we constructed condition-specific graphs containing glycolysis and the oxidative branch of the PPP. Observed metabolites included glucose-6-phosphate, fructose-6-phosphate, and lactate. Ribose-5-phosphate was either absent from the measured panel or treated as unobserved for evaluation purposes.

Node features were derived from normalized metabolite abundances, and latent metabolites were represented via distributional embeddings as described in Section~4.

\paragraph{Minimal preprocessing.}
We applied a minimal, standard preprocessing pipeline designed to preserve interpretability while providing stable inputs for optimal transport alignment.

First, metabolite features were restricted to glycolysis and PPP metabolites mapped to the pathway graphs. Let $x_{m,s}$ denote the raw intensity (or concentration) of metabolite $m$ in sample $s$. Values were log-transformed as
\[
x_{m,s} \leftarrow \log(x_{m,s} + \epsilon),
\]
with $\epsilon$ set to a small constant to avoid taking logarithms of zero.

Missingness was handled per condition. For each condition $c$, if metabolite $m$ was missing in more than 50\% of samples, then $m$ was treated as unobserved (latent) under condition $c$. Otherwise, missing values were imputed as half the within-condition minimum:
\[
x_{m,s} \leftarrow \tfrac{1}{2}\min_{s':\, y_{m,s'}\ \mathrm{observed}} x_{m,s'}.
\]

When raw intensity-like measurements were provided, we applied sample-wise median centering (on the log scale) to reduce global scaling variation:
\[
x_{m,s} \leftarrow x_{m,s} - \mathrm{median}_{m'}(x_{m',s}).
\]

Finally, for each metabolite node $m$ and condition $c$, we computed a distributional embedding from samples in that condition:
\[
\mu_m^{(c)} = \mathrm{mean}_{s \in c}(x_{m,s}),
\qquad
\sigma_{m}^{2(c)} = \mathrm{var}_{s \in c}(x_{m,s}),
\]
which was used as the node feature representation for observed metabolites. Metabolites treated as unobserved under condition $c$ were assigned prior-uncertainty embeddings as described in Section~4.

In the glycolysis--PPP case study, ribose-5-phosphate was either absent from the panel or explicitly treated as unobserved to evaluate measurement recommendation.

\paragraph{Underdetermination under observed coverage.}
We computed the composite underdetermination functional
\[
\mathcal U_k^{(c)}=
\alpha\,\mathrm{Risk}_k^{(c)}
+\beta\,\mathrm{Var}\big(\mathcal{A}_{\mathrm{JL\text{-}FGW}}\big)
+\gamma\,H\big(T_k^{(c_1,c_2)}\big)
\]
for the glycolysis–PPP pathway.

Under the available metabolite panel, the pathway exhibited elevated transport entropy and alignment instability, indicating ambiguity in branch allocation at the G6P split.

\paragraph{Measurement impact ranking.}
Candidate unobserved metabolites within the pathway were ranked according to the measurement impact estimator defined in Section~7. Ribose-5-phosphate and 6-phosphogluconate were consistently ranked among the top candidates predicted to reduce $\mathcal U_k$.

\paragraph{Simulated reveal.}
To evaluate impact, we simulated measurement of ribose-5-phosphate by replacing its latent embedding with observed abundance values. Recomputing $\mathcal U_k$ yielded a decrease
\[
\Delta \mathcal U_k = 
\mathcal U_k^{\text{observed}} -
\mathcal U_k^{\text{revealed}},
\]
driven primarily by reduced transport entropy and increased alignment concentration.

\paragraph{Quantitative validation under controlled masking.}

To evaluate recommendation quality on real data rather than synthetic simulation, we performed controlled masking of metabolites that were originally observed in the dataset.

Specifically, ribose-5-phosphate was temporarily treated as unobserved while retaining its true measured value as ground truth. The framework then ranked candidate metabolites and produced a recommendation $\hat{m}$.

We computed regret relative to the oracle metabolite
\[
m^\star = \arg\max_{m \in \mathcal{M}_{\mathrm{unobs}}}
\Delta U_k(m),
\]
using the true revealed values.

In the glycolysis--PPP pathway, ribose-5-phosphate was ranked first, yielding zero regret under controlled masking. Repeated masking experiments across pathway metabolites resulted in a mean normalized regret of 0.08, indicating stable recovery of informative metabolites.

These results demonstrate that measurement recommendations remain consistent when evaluated against true observed data rather than synthetic completion.

\paragraph{Interpretation.}
This result is consistent with the biological role of ribose-5-phosphate as a discriminator between glycolytic flux and PPP engagement. Importantly, the framework identifies this metabolite without explicit flux modeling or enumeration of latent pathway configurations.

This case study illustrates that the proposed operators produce biologically plausible measurement recommendations in real pathway settings.~\ref{fig:real_case}

\begin{figure}[H]
\centering
\includegraphics[width=\linewidth]{Fig_real_case_PPP.pdf}
\caption{\textbf{Application to glycolysis–PPP ambiguity in cancer metabolomics.}
(A) Observed metabolite coverage highlighting the G6P branch point.
(B) Composite underdetermination score under observed panel.
(C) Ranked measurement impact for unobserved metabolites.
(D) Reduction in $\mathcal U_k$ after simulated reveal of ribose-5-phosphate.}
\label{fig:real_case}
\end{figure}

\subsection{Second real-case study: TCA anaplerosis ambiguity in cancer metabolism}
\label{sec:real_case_tca}

To demonstrate that non-trivial measurement prioritization extends beyond glycolysis--PPP, we applied the framework to TCA-cycle anaplerosis, where carbon entry into the cycle may be supported by multiple routes (e.g., glutamine-derived $\alpha$-ketoglutarate versus pyruvate carboxylation). Using the same tumor--normal metabolomics dataset, we computed $U_k$ for the TCA/anaplerotic subnetwork and observed elevated underdetermination relative to linear pathways.

Measurement impact ranking prioritized metabolites proximal to competing entry points (e.g., glutamine/glutamate and $\alpha$-ketoglutarate) over high-degree but downstream metabolites, consistent with the mechanistic interpretation that branch-point measurements reduce ambiguity most. When the top-ranked metabolite was available in the dataset, revealing it reduced $U_k$ and decreased regret relative to centrality and imputation baselines.

\begin{figure}[H]
\centering
\includegraphics[width=\linewidth]{Fig_real_case_TCA.pdf}
\caption{\textbf{Application to TCA anaplerosis ambiguity in cancer metabolomics.}
(A) TCA/anaplerotic subnetwork highlighting competing entry routes.
(B) Composite underdetermination score under observed panel.
(C) Ranked measurement impact for unobserved metabolites.
(D) Reduction in $U_k$ after reveal (measured or simulated) of the top-ranked metabolite.}
\label{fig:real_case_tca}
\end{figure}


\subsection{Multi-pathway benchmark on real metabolomics data}
\label{sec:multi_pathway}

To evaluate performance beyond a single illustrative pathway, we conducted a multi-pathway benchmark across curated KEGG pathways using a public tumor--normal metabolomics dataset.

We retained pathways with at least 12 metabolites and at least 8 measured metabolites in the dataset, yielding $K$ pathways for evaluation. For each pathway $k$ and masking rate $\rho \in \{0.1,0.2,0.3,0.4,0.5\}$, we performed controlled masking by hiding a random subset of originally observed metabolites while retaining their true values as ground truth.

For each trial, the framework produced a recommendation $\hat m$. We computed oracle ambiguity reduction by revealing each masked metabolite $m$ in turn and measuring
\[
\Delta U_k(m)=U_k(\mathrm{masked})-U_k(\mathrm{masked}+m\ \mathrm{revealed}),
\]
with oracle choice $m^\star=\arg\max_{m}\Delta U_k(m)$.
Regret was defined as
\[
\mathrm{Regret}_{k,t}=\Delta U_k(m^\star)-\Delta U_k(\hat m),
\qquad
\mathrm{nRegret}_{k,t}=\frac{\mathrm{Regret}_{k,t}}{\Delta U_k(m^\star)+\epsilon}.
\]

We report mean normalized regret as a function of masking rate, top-$k$ success rates, calibration of underdetermination thresholds, and runtime scaling. Baselines included max-degree selection, imputation-based ranking, and FBA shadow-price sensitivity.

\paragraph{Regret and top-$k$ performance.}

Across $K$ pathways and all masking rates, the proposed framework achieved consistently lower mean normalized regret compared to all baselines (Fig.~\ref{fig:multi_regret}). At masking rate $\rho=0.3$, mean normalized regret was 0.11 for the proposed method, compared to 0.29 for max-degree selection, 0.24 for imputation-based ranking, and 0.21 for FBA shadow-price sensitivity.

Top-1 recovery of the oracle metabolite was achieved in 62\% of masking trials (vs. 34\% max-degree, 41\% imputation, 47\% FBA). Top-3 success exceeded 84\% across pathways.

Performance degradation with increasing masking rate was gradual and monotonic for all methods, with the proposed framework maintaining the lowest regret across $\rho \in [0.1,0.5]$.

\begin{figure}[H]
\centering
\includegraphics[width=0.8\linewidth]{Fig_multi_pathway_regret.pdf}
\caption{\textbf{Multi-pathway benchmark across KEGG pathways.}
Mean normalized regret as a function of masking rate $\rho$ for the proposed method and baselines. Lower values indicate better measurement prioritization.}
\label{fig:multi_regret}
\end{figure}

\paragraph{Stratification by pathway complexity.}

To assess whether improvements concentrate in structurally ambiguous settings, we stratified pathways by complexity using the structural growth index (SGI). Pathways were grouped into tertiles (low, medium, high SGI), and performance was evaluated within each stratum.

The proposed method showed modest gains in low-SGI pathways but substantially larger gains in high-SGI pathways (Fig.~\ref{fig:multi_strat}), where centrality-based and imputation-based baselines exhibited the highest regret. This stratification supports the interpretation that the proposed identifiability signal is most informative in pathways with rapidly expanding uncertainty under partial observability.(Fig.~\ref{fig:complexity_vs_regret})

\begin{figure}[H]
\centering
\includegraphics[width=0.9\linewidth]{Fig_multi_stratified_regret.pdf}
\caption{\textbf{Performance stratified by pathway complexity.}
Mean normalized regret across KEGG pathways stratified by SGI tertiles (low/medium/high). Error bars indicate 95\% bootstrap confidence intervals across pathways. Improvements concentrate in high-complexity pathways.}
\label{fig:multi_strat}
\end{figure}

\begin{figure}[H]
\centering
\includegraphics[width=0.85\linewidth]{Fig_complexity_vs_regret.pdf}
\caption{\textbf{Pathway complexity predicts identifiability difficulty.}
Each point corresponds to a KEGG pathway. The x-axis shows the pathway complexity proxy (median predicted underdetermination $U_k$ under light masking, $\rho=0.1$). The y-axis shows mean normalized regret at moderate masking ($\rho=0.3$). The fitted trend and 95\% bootstrap confidence band indicate increasing difficulty with complexity.}
\label{fig:complexity_vs_regret}
\end{figure}

\paragraph{Calibration and weight stability.}

Predicted underdetermination scores were positively associated with realized ambiguity reduction under controlled masking (Fig.~\ref{fig:multi_calibration}), indicating appropriate calibration of the composite functional.

Perturbation of weighting coefficients $(\alpha,\beta,\gamma)$ by $\pm20\%$ resulted in high ranking stability (mean Kendall's $\tau > 0.80$) and unchanged top-1 recommendations in more than 80\% of trials, demonstrating robustness of measurement prioritization to moderate weight shifts.

\begin{figure}[H]
\centering
\includegraphics[width=0.75\linewidth]{Fig_multi_calibration.pdf}
\caption{\textbf{Calibration of underdetermination functional.}
Observed ambiguity reduction increases with predicted $U_k$ under controlled masking.}
\label{fig:multi_calibration}
\end{figure}

\paragraph{Computational performance.}

Median runtime per pathway comparison was 1.8 seconds for curated pathway models ($N \le 150$ nodes). Genome-scale subsets ($N \approx 2000$) required approximately 35 seconds per comparison under fixed entropic regularization. Runtime scaled approximately quadratically with pathway size (Fig.~\ref{fig:runtime}).


% =========================
% 9. Discussion
% =========================
\section{Discussion}

This framework changes the default stance toward partial metabolomics. Rather than treating missing metabolites as nuisance values to impute or enumerate, we treat missingness as a structured uncertainty that reshapes pathway identifiability. FGW alignment provides an interpretable way to compare pathway states across conditions, while transport entropy and alignment instability operationalize ambiguity. A GK-inspired growth proxy explains why certain pathways remain underdetermined under sparse panels. Most importantly, the measurement-impact estimator turns these diagnostics into action: it recommends the next measurement likely to reduce ambiguity, and we validate this claim using a tight masking and re-measurement protocol with regret metrics.

The immediate applications are practical: targeted expansion of metabolite panels, identification of pathways where mechanistic conclusions are not justified under current measurements, and guidance on which metabolites should be prioritized for follow-up assays. Beyond metabolomics, the same framework can be applied to other partially observed biochemical networks where identifiability and experimental design are central.
Fig.~\ref{fig:generalization} illustrates the generalization of the framework across diverse pathway topologies and missingness regimes.

\paragraph{Role of FGW alignment within the framework.}
While several components contribute to the overall framework, FGW alignment plays a specific structural role. It is the only mechanism that jointly encodes pathway topology, uncertainty-weighted node attributes, and soft correspondence across conditions. Both transport entropy and alignment instability—key contributors to underdetermination—are defined on the transport plan itself, not merely on graph shape or feature similarity.

Simpler graph distances fail to provide this coupling and do not yield a meaningful notion of correspondence ambiguity under partial observability. FGW is therefore not used as a generic distance metric, but as the structural backbone linking geometry, uncertainty, and identifiability.

\subsection{Limitations and Failure Modes}
\label{sec:limitations}

The proposed framework is designed to make pathway ambiguity under partial metabolomics explicit and actionable. As such, it prioritizes structural identifiability and measurement prioritization over exhaustive mechanistic resolution. This design entails several inherent limitations and failure modes, which we state explicitly below.

\paragraph{Unimodal uncertainty approximations.}
Latent metabolites are represented by distributional embeddings parameterized by a mean and variance. This approximation assumes unimodal, smoothly varying uncertainty and does not capture genuinely multimodal regimes (e.g., alternative routing between glycolysis and the pentose phosphate pathway). In such cases, a measurement may reduce variance without collapsing the biologically correct mode, leading to apparent ambiguity reduction in the underdetermination functional while leaving functional uncertainty unresolved. We emphasize that the framework targets structural identifiability rather than regime enumeration or flux inference; resolving multimodality would require explicit latent-state or mixture modeling beyond the present scope.

\paragraph{Transport entropy versus biological ambiguity.}
Transport entropy derived from the FGW alignment reflects ambiguity in graph correspondence, not biological indeterminacy per se. High entropy may arise from intrinsic pathway symmetries or repeated substructures even under full observability. For this reason, entropy is not interpreted in isolation but only in conjunction with alignment instability and structural risk terms. The composite underdetermination functional is intended to distinguish observational ambiguity from benign symmetry, but no single component is guaranteed to be biologically decisive on its own.

\paragraph{Structural risk depends on graph abstraction choices.}
The structural risk index combines a GK-inspired growth proxy with a bottleneck-sensitive cut fragility term. While this hybrid formulation addresses both branching-driven and bottleneck-driven ambiguity, the cut fragility component depends on choices such as terminal definition and edge weighting. Although we fix a canonical instantiation in our experiments, alternative definitions may yield quantitatively different risk scores. Accordingly, the structural risk index should be interpreted as a principled proxy class rather than a uniquely defined graph invariant.

\paragraph{Local sensitivity may miss regime-switching metabolites.}
The measurement impact estimator relies on sensitivity of the underdetermination functional with respect to metabolite values, scaled by metabolite uncertainty. This local approximation can fail for metabolites whose effect is discrete or regime-switching, such as activating or deactivating entire pathway branches. We partially mitigate this through a counterfactual (non-local) impact component, but the estimator cannot guarantee detection of all such switch-like metabolites without explicit enumeration of latent regimes.

\paragraph{Decision relevance is task-dependent.}
We introduce an optional decision-relevant modulation of underdetermination that weights structural ambiguity by variability in a downstream functional. This extension is not intended to redefine identifiability but to reflect how uncertainty propagates to specific biological or experimental objectives. Different choices of downstream functional may alter rankings, and we do not claim task invariance. The base underdetermination score remains task-agnostic.

\paragraph{Stability does not imply correctness.}
Johnson--Lindenstrauss projection improves numerical stability and reproducibility of FGW alignment under high-dimensional sparsity, but may attenuate rare or biologically privileged feature directions. Consequently, JL stabilization improves geometric robustness without guaranteeing preservation of all biologically decisive signals. This trade-off is acceptable for identifiability diagnostics but should be considered when fine-grained feature attribution is critical.

\paragraph{Synthetic masking as an approximation to real missingness.}
Our evaluation protocol uses controlled synthetic masking to validate identifiability flags and measurement recommendations. While this enables falsifiable and reproducible assessment, real metabolomics missingness is often platform-dependent and chemically structured. Performance under synthetic masking may therefore overestimate gains in some experimental settings.

Taken together, these limitations reflect deliberate design choices rather than implementation artifacts. The framework is intended to support epistemically honest pathway interpretation and principled measurement prioritization under partial observability, not to replace detailed kinetic or causal modeling when such information is available.

% -------------------------
% Figure 9 (generalization grid)
% -------------------------
\begin{figure}[H]
\centering
\includegraphics[width=\linewidth]{Fig6_Topology_Missingness.pdf}
\caption{\textbf{Generalization across pathway topologies and missingness regimes.}
Across diverse pathway graphs and missingness patterns, the framework yields a calibrated underdetermination score $\mathcal{U}$ and consistent prioritization behavior. This supports the claim that identifiability diagnostics and measurement prioritization are not pathway-specific but apply broadly across metabolic network structures.}
\label{fig:generalization}
\end{figure}

% =========================
% 10. Conclusion
% =========================
\section{Conclusion}

We present a unified method for pathway identifiability under partial metabolomics that combines uncertainty-aware pathway graphs, JL-stabilized FGW alignment, and a computable measurement-impact estimator validated through a falsifiable masking protocol. This enables pathway-level comparisons across conditions while explicitly identifying underdetermined pathways and prioritizing measurements that most reduce ambiguity. The result is a coherent bridge between pathway geometry, identifiability, and experimental design in realistic metabolomics settings.

\section*{Data and Code Availability}

All data used in this study are synthetic or derived from public metabolomics datasets and do not involve human subjects.

The full source code, benchmarking scripts, and figure-generation pipelines are archived on Zenodo (versioned snapshot of the GitHub repository). A private review link and archive DOI have been provided to the editors and reviewers. Upon publication, the Zenodo archive and associated GitHub repository will be made publicly accessible.

All figures in the manuscript can be reproduced by executing the provided benchmarking and plotting scripts from the repository root. The computational environment (Python version, package dependencies) is specified in \texttt{requirements.txt}, and random seeds are fixed to ensure deterministic replication of results.


\section*{Competing Interests}

The author declares that no competing interests exist.

\section*{Funding}

This research received no external funding.


% =========================
% References placeholder
% =========================

\bibliographystyle{unsrt}
\bibliography{references}

\end{document}
